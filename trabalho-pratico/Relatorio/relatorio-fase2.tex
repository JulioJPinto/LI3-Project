\documentclass{article}

\usepackage[portuguese]{babel}

\usepackage[a4paper]{geometry}
\usepackage{pbox}
\usepackage{caption, booktabs}
\usepackage{makecell}
\usepackage{cellspace}
\usepackage{lipsum}  


\title{Relatório da Fase 2 - Grupo 03}
\author{Francisco Macedo Ferreira (A100660)\\Júlio José Medeiros Pereira Pinto (A100742)}

\begin{document}  
    \maketitle
    Multihreading entre drivers e users não foi implementado porque loads de drivers e users são rápidos.
    O que impacta é o load de rides.
    Pode ser feito multithreading no loading de rides, mas não é prioridade.

    Lazy loading para implementar.
    Unit tests para implementar.

    Csv generator mt fixe.

    Data a usar 1 int com bitwise operations em vez de 3.

    \section{Introdução}
    \lipsum[1]
    
    \section{Desenvolvimento}
    O projeto 
    \subsection{Ajustes sobre a fase 1}
        \subsubsection{Ajustes de críticas de docentes}
            TODO: dizer que dividimos os catálogos em 3. e não me lembro mais

        \subsubsection{Ajustes de fator escala (Dataset maior)}
            Da forma como tínhamos o projeto implementado, quando 
            usado o dataset maior, o programa mesmo assim continuava
            a performar dentro dos limites de tempo com grandes 
            margens. As estruturas de dados e estratégias para
            a resolução de queries foram então adequadas, pelo que
            não alteramos a implementação das queries já feitas
            (1, 2, 3, 4 e 5). As previsões de possíveis alterações
            mencionadas na fase 1 não foram necessárias de serem
            implementadas, exceto o \emph{lazy loading} que foi
            útil no modo interativo (será abordado mais abaixo).
        
    \subsection{Pipeline}
        


    

\end{document}
