\documentclass{article}

\usepackage[portuguese]{babel}

\usepackage[a4paper]{geometry}
\usepackage{pbox}
\usepackage{caption, booktabs}
\usepackage{makecell}
\usepackage{cellspace}
\usepackage{lipsum}  


\title{Relatório da Fase 1 - Grupo 03}
\author{Francisco Macedo Ferreira (A100660)\\Hugo ... Ramos (A100644)\\Júlio José Medeiros Pereira Pinto (A100742)}

% Document
\begin{document}

    \maketitle

    \section{Introdução}

    Esta é a introdução.

    \section{Desenvolvimento}

        Este é o desenvolvimento.

        \subsection{Pipeline atual}
            O nosso pipeline funciona da seguinte maneira.
            A main recebe primeiro os \emph{paths} do \emph{dataset} e do \emph{input}.
            A seguir esta manda os ficheiros para a função \emph{read_file} e aplica a
            cada linha de cada ficheiro do dataset e retorna a respetiva estrutura de 
            dados devidamente \emph{parsed}.
            Estas estruturas são enviadas para um catálogo onde será feito o indexamento
            em \emph{HashTables} e \emph{arrays} das respetivas estruturas, através do
            uso de funções como as \emph{register} no \emph{catalog.c}. No fim da criação
            do catálogo este será ordenada, através da \emph{notifiy_stop_registering}
            de maneira a tornar as queries ainda mais efecientes.
            
        \subsection{Estratégias Seguidas}

        Para a primeira fase do projeto foram implementadas as queries 1, 2, 3, 4 e 5.

        \subsubsection{Query 1}
            Para acesso rápido a perfis pelo \emph{id's} de condutores ou 
            \emph{username's} de utilizadores foram utilizadas duas 
            \emph{HashTables} (uma para indexar os perfis de condutores e outra 
            para perfis de utilizadores) com \emph{key id e username} e 
            \emph{value Driver* e User*} respetivamente. Esses \emph{HashTables}
            são populados conforme a leitura dos ficheiros {.csv} e informações
            como número de viagens, soma total de avaliações (para poder ser 
            calculado a média), total gasto/auferido eram calculados e guardados 
            na estrutura de dados dos utilizadores e dos condutores\footnote{Pode 
            ser uma possível quebra de encapsulamento. Ainda aguardamos a resposta
            do docente sobre esse quesito.}. A decisão de buscar informações de
            utilizadores ou condutores é feita pela confirmação se o primeiro 
            argumento da query é um número ou não: se for um número busca-se por
            condutores (\emph{id}), se não busca-se por utilizadores (\emph{username}).
            \\
            \\Com isto é possível um acesso em tempo constante a essas
            informações, com o custo da leitura dos ficheiros ser um pouco maior
            (devido ao calculo necessário de \emph{hashes} para cada \emph{key} 
            e não só). Como na segunda fase o número de perfis irá aumentar
            exponencialmente e é esperado múltiplos acessos a estas informações, 
            seja de várias queries destas ou do modo interativo ainda a 
            implementar, o custo de leitura dos ficheiros superior é muito 
            justificado.
        \subsubsection{Query 2}
            Na query 2, para rápido acesso aos condutores com maior média foi 
            feito \emph{sorting} da \emph{array} (presente no catálogo) de 
            condutores, conforme a sua média, no fim da leitura dos ficheiros.
            Isto poderá ser otimizado a fazer com que o \emph{sorting} seja
            \emph{lazy} (será abordado este tópico mais tarde).
            Durante a execução da query basta obter os N primeiros 
            condutores da \emph{array} e temos a execução em práticamente em
            tempo constante (agora é copiado os N elementos para uma array, mas
            poderá ser otimizado).
        \subsubsection{Query 3}
            Para a query 3, tal como na query 2, foi feito um \emph{sorting} da
            \emph{array} (presente no catálogo) de utilizadores, de maneira a 
            otimizar o acesso aos users com maior distância total percorrida.
            Tal como na query 2, o método de \emph{sorting} poderá ser otimizado
            da mesma maneira.
            Durante a execução destea query o tempo de execução será praticamente
            contstante, porém poderá vir a ser otimizado a maneira como os N
            primeiros utilizadores são guardados.        
        \subsubsection{Query 4}
            Nesta query, o preço médio das viagens numa determinada cidade é 
            calculada (em tempo linear conforme o número de viagens por cidade)
            durante a execução da query. Durante a leitura das viagens,
            é inserido a viagem conforme a sua cidade numa 
            \emph{HashTable (key: cidade, value: Array de Ride*)}. 
            O preço médio não é pré-calculado, pois o cálculo desta é 
            relativamente rápido devido às viagens já estarem separadas por cidade
            e é expectável que só se aceda a este valor uma vez, por isso
            guardá-lo será desnecessário.
            Esta \emph{HashTable} já existe devido à query 7 que necessitará de
            acesso rápido a viagens conforme a sua cidade. 
            \\
            \\Futuramente, conforme a expansão do dataset, esta implementação 
            poderá ter que ser reformulada, por causa do cálculo em execução mas
            para já tivemos bom desempenho com a atual.
        
        \subsubsection{Query 5}
            \newcommand{\param}[1]{$<${#1}$>$}
            Já na query 5, no fim da leitura das viagens é feito \emph{sorting}
            das viagens pela sua data, por ordem crescente. Com isto, assumindo 
            que \param{data A} e \param{data B} são os argumentos da query, 
            basta aceder ao primeiro elemento a partir do qual \param{data A} é 
            menor ou igual do que a data desse elemento. A partir daí, podemos
            percorrer a lista até encontrar uma viagem que a sua data seja
            maior que \param{data B}, acumulando o preço das viagens para no fim
            calcular a sua média. O tal primeiro elemento é encontrado com
            uma implementação semelhante à \textbf{std::lower\_bound} de \emph{C++},
            usando \emph{binary search}. Isto é possível devido à lista estar
            organizada pela data das viagens.
            \\
            \\Para evitar percorrer a lista, podia ter sido pré-calculado uma \emph{array}
            em que cada indíce tinha o somatório de preços para trás desse indíce
            e o preço médio era calculado subtraindo o \emph{upper\_bound}
            com o \emph{lower\_bound} do \emph{range} das datas e dividindo pelo 
            número de elementos entre eles. Essa ideia foi rápidamente descartada
            devido às datas dos argumentos serem relativamente perto, portanto
            são poucas as viagens a iterar. Esta solução também iria aumentar 
            consideravelmente o tempo de leitura dos ficheiros.

        \newpage
        \subsection{Análise de desempenho}
            Comparando o desempenho da execução das queries 1, 2, 3, 4 e 5 para 
            os ficheiros de input do conjunto de testes expandido das pastas 
            \emph{tests1} e \emph{tests2} (excluindo as queries não 
            implementadas) temos na Tabela 2 os resultados conforme as 
            especificações dos computadores na Tabela 1.
            \\
            \\O programa foi compilado com as flags \emph{-O2 -flto}.
            
            \begin{table}[hbt!]
                \centering
                \begin{tabular}{|*{4}{c|}}
                    \hline
                    & \thead{PC 1}&\thead{PC 2}&\thead{PC 3}\\
                    \hline
                    CPU        & M1 Pro 8-core (6 perf. e 2 ef.) & - & - \\
                    RAM        & 16GB RAM LPDDR5                 & - & - \\
                    Disco      & 500GB NVME                      & - & - \\
                    OS         & MacOS Ventura 13.0.1            & - & - \\
                    Compilador & Apple Clang 14.0.0              & - & - \\
                    \hline
                \end{tabular}
                \caption{Performances em diferentes PCs}
            \end{table}

            \begin{table}[hbt!]
                \centering
                \begin{tabular}{|*{4}{c|}}
                    \hline
                    & \thead{PC 1}&\thead{PC 2}&\thead{PC 3}\\
                    \hline
                    tests 1.txt & - & - & - \\
                    tests 2.txt & - & - & - \\
                    \hline
                \end{tabular}
                \caption{Tempos de execução em diferentes PCs}
            \end{table}

        \subsection{Possíveis otimizações e futuras limitações}
        \lipsum[1]
    \section{Conclusão}
    \lipsum[1]

\end{document}