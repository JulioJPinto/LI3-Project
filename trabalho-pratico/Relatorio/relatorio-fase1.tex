\documentclass{article}

\usepackage[portuguese]{babel}

\usepackage[a4paper]{geometry}
\usepackage{pbox}
\usepackage{caption, booktabs}
\usepackage{makecell}
\usepackage{cellspace}
\usepackage{lipsum}  


\title{Relatório da Fase 1 - Grupo 03}
\author{Francisco Macedo Ferreira (A100660)\\Hugo ... Ramos (A100644)\\Júlio José Medeiros Pereira Pinto (A100742)}

% Document
\begin{document}

    \maketitle

    \section{Introdução}

    Este relatório tem como intuito abordar a primeira fase do projeto da UC Laboratórios 
    de Informática III do ano letivo 2022/2023.\\
    Nesta fase como objetivos o trabalho necessitava a implementação de \emph{parsing} dos
    dados de entrada, funcionar através do modo de operação \emph{batch}, realizar pelo menos um terço das \emph{queries} (pelo menos 3 \emph{queries}) e deverá também considerar que o tamanho dos ficheiros de entrada (\emph{users.csv, drivers.csv} e \emph{rides.csv}), em linhas, terá 100.000, 10.000 e 1.000.000, respetivamente.
    Ainda mais, no projeto, em geral, existe uma necessidade do uso de conceitos de encapsulamento e modularidade.
    \\Este relatório irá abrangir as decisões tomadas pelo grupo tal como o método de raciocínio para desenvolvimento do mesmo.
    

    \section{Desenvolvimento}

        \subsection{Pipeline atual}
            A nossa pipeline foi alvo de bastante discussão entre nós, porém conseguimos chegar a um consenso
            com a pipeline atual. Neste momento a nossa pipeline funciona da seguinte maneira. A nossa main recebe dois argumentos, argumentos estes que serãos os \emph{paths} para os ficheiros necessários, a pasta do \emph{dataset}, onde se encontram os .csv, e o ficheiro de \emph{input}. De seguida, este abre os nosso ficheiros (ficheiros que se encontram na pasta \emph{dataset}) criando \emph{pointers} para os mesmos, assim podendo realizar o \emph{parsing} destes. No \emph{parsing} a função irá receber um argumento de \emph{stream} (o ficheiro que irá dar \emph{parse}) , uma função para o \emph{parse} linha a linha do ficheiro (Função que irá criar a estrutura que pretendemos através da linha), uma função \emph{register} que irá enviar a estrutura criada para um catálogo e por último um argumento, que do ponto de vista do código não sabemos o tipo do mesmo, porém podemos deduzir que será o catálogo para onde enviaremos as estruturas criadas pelo \emph{parser}.
            \\ Após finalizar o \emph{parsing} dos ficheiros, este irá ordenar o nosso catálogo da maneira mais útil para a realização das queries (Irá ordenar o \emph{array} de condutores por score, o \emph{array} de utilizadores por distância total percorrida e o \emph{array} de viagens por data). Desta maneira a tornar as queries mais eficientes quando forem a ser realizadas.
            \\Por último esta irá ler o ficheiro de \emph{input} e irá dar parse ao mesmo, realizando então as queries, através do nosso \emph{query manager} que gere as funções para a realização das queries de maneira a utilizar a query certa dependendo do \emph{input} passado. Assim, após libertarmos a memória alocada, concluindo o nosso pipeline atual.
            
        \subsection{Estratégias Seguidas}

            Para a primeira fase do trabalho decidimos implementar as bases do projeto, como o parser e definir as estruturas dos dados que utilizaremos ao longo do projeto ( Como o catálogo para guardar os dados e os \emph{users, drivers} e \emph{rides} de forma a guardar os respetivos campos dos utilizadores, condutores e viagens). Ainda mais implementamos as \emph{queries} 1, 2, 3, 4 e 5. 
        \subsubsection{Parser}
            De maneira a tornar a leitura e o \emph{parsing} dos ficheiros de
            entrada relativamente eficiente, decidimos percorrer o ficheiro linha a linha.
            Sempre que este lê uma linha, de acordo com a função de \emph{parsing} passada
            nesta, este irá dar \emph{parse} à linha que recebe, transformando o na respetiva estrutura de dados (\emph{users, drivers} e \emph{rides}). Ainda mais este insere a estrutura de dados no respetivo array do catálogo (argumento passado na chamada da função porém do ponto de vista do código argumento o qual não sabemos o tipo).
            \\Futuramente poderemos otimizar o código de maneira a tornar a leitura do ficheiro mais rápido, visto que o tamanho dos ficheiros também  irá aumentar, tal como poderemos otimizar o registo das estruturas de dados para o catálogo.
        \subsubsection{Query Manager}
            Face a necessidade da gestão das queries, implementamos um \emph{Query Manager}.
            
        \subsubsection{Query 1}
            Para acesso rápido a perfis pelo \emph{id's} de condutores ou 
            \emph{username's} de utilizadores foram utilizadas duas 
            \emph{HashTables} (uma para indexar os perfis de condutores e outra 
            para perfis de utilizadores) com \emph{key id e username} e 
            \emph{value Driver* e User*} respetivamente. Esses \emph{HashTables}
            são populados conforme a leitura dos ficheiros {.csv} e informações
            como número de viagens, soma total de avaliações (para poder ser 
            calculado a média), total gasto/auferido eram calculados e guardados 
            na estrutura de dados dos utilizadores e dos condutores\footnote{Pode 
            ser uma possível quebra de encapsulamento. Ainda aguardamos a resposta
            do docente sobre esse quesito.}. A decisão de buscar informações de
            utilizadores ou condutores é feita pela confirmação se o primeiro 
            argumento da query é um número ou não: se for um número busca-se por
            condutores (\emph{id}), se não busca-se por utilizadores (\emph{username}).
            \\
            \\Com isto é possível um acesso em tempo constante a essas
            informações, com o custo da leitura dos ficheiros ser um pouco maior
            (devido ao calculo necessário de \emph{hashes} para cada \emph{key} 
            e não só). Como na segunda fase o número de perfis irá aumentar
            exponencialmente e é esperado múltiplos acessos a estas informações, 
            seja de várias queries destas ou do modo interativo ainda a 
            implementar, o custo de leitura dos ficheiros superior é muito 
            justificado.
        \subsubsection{Query 2}
            Na query 2, para rápido acesso aos condutores com maior média foi 
            feito \emph{sorting} da \emph{array} (presente no catálogo) de 
            condutores, conforme a sua média, no fim da leitura dos ficheiros.
            Isto poderá ser otimizado a fazer com que o \emph{sorting} seja
            \emph{lazy} (será abordado este tópico mais tarde).
            Durante a execução da query basta obter os N primeiros 
            condutores da \emph{array} e temos a execução em práticamente em
            tempo constante (agora é copiado os N elementos para uma array, mas
            poderá ser otimizado).
        \subsubsection{Query 3}
            Para a query 3, tal como na query 2, foi feito um \emph{sorting} da
            \emph{array} (presente no catálogo) de utilizadores, de maneira a 
            otimizar o acesso aos users com maior distância total percorrida.
            Tal como na query 2, o método de \emph{sorting} poderá ser otimizado
            da mesma maneira.
            Durante a execução desta query o tempo de execução será praticamente
            constante, porém poderá vir a ser otimizado a maneira como os N
            primeiros utilizadores são guardados.        
        \subsubsection{Query 4}
            Nesta query, o preço médio das viagens numa determinada cidade é 
            calculada (em tempo linear conforme o número de viagens por cidade)
            durante a execução da query. Durante a leitura das viagens,
            é inserido a viagem conforme a sua cidade numa 
            \emph{HashTable (key: cidade, value: Array de Ride*)}. 
            O preço médio não é pré-calculado, pois o cálculo desta é 
            relativamente rápido devido às viagens já estarem separadas por cidade
            e é expectável que só se aceda a este valor uma vez, por isso
            guardá-lo será desnecessário.
            Esta \emph{HashTable} já existe devido à query 7 que necessitará de
            acesso rápido a viagens conforme a sua cidade. 
            \\
            \\Futuramente, conforme a expansão do dataset, esta implementação 
            poderá ter que ser reformulada, por causa do cálculo em execução mas
            para já tivemos bom desempenho com a atual.
        
        \subsubsection{Query 5}
            \newcommand{\param}[1]{$<${#1}$>$}
            Já na query 5, no fim da leitura das viagens é feito \emph{sorting}
            das viagens pela sua data, por ordem crescente. Com isto, assumindo 
            que \param{data A} e \param{data B} são os argumentos da query, 
            basta aceder ao primeiro elemento a partir do qual \param{data A} é 
            menor ou igual do que a data desse elemento. A partir daí, podemos
            percorrer a lista até encontrar uma viagem que a sua data seja
            maior que \param{data B}, acumulando o preço das viagens para no fim
            calcular a sua média. O tal primeiro elemento é encontrado com
            uma implementação semelhante à \textbf{std::lower\_bound} de \emph{C++},
            usando \emph{binary search}. Isto é possível devido à lista estar
            organizada pela data das viagens.
            \\
            \\Para evitar percorrer a lista, podia ter sido pré-calculado uma \emph{array}
            em que cada indíce tinha o somatório de preços para trás desse indíce
            e o preço médio era calculado subtraindo o \emph{upper\_bound}
            com o \emph{lower\_bound} do \emph{range} das datas e dividindo pelo 
            número de elementos entre eles. Essa ideia foi rápidamente descartada
            devido às datas dos argumentos serem relativamente perto, portanto
            são poucas as viagens a iterar. Esta solução também iria aumentar 
            consideravelmente o tempo de leitura dos ficheiros.

        \newpage
        \subsection{Análise de desempenho}
            Comparando o desempenho da execução das queries 1, 2, 3, 4 e 5 para 
            os ficheiros de input do conjunto de testes expandido das pastas 
            \emph{tests1} e \emph{tests2} (excluindo as queries não 
            implementadas) temos na Tabela 2 os resultados conforme as 
            especificações dos computadores na Tabela 1.
            \\
            \\O programa foi compilado com as flags \emph{-O2 -flto}.
            
            \begin{table}[hbt!]
                \centering
                \begin{tabular}{|*{4}{c|}}
                    \hline
                    & \thead{PC 1}&\thead{PC 2}&\thead{PC 3}\\
                    \hline
                    CPU        & M1 Pro 8-core (6 perf. e 2 ef.) & Intel i7-8550U (8)  & - \\
                    RAM        & 16GB RAM LPDDR5                 & 8GB RAM DDR4 2400MHz  & - \\
                    Disco      & 500GB NVME                      & 500GB NVME & - \\
                    OS         & MacOS Ventura 13.0.1            & ArcoLinux Kernel: 6.0.9-arch-1 & - \\
                    Compilador & Apple Clang 14.0.0              & GCC 12.2.0  & - \\
                    \hline
                \end{tabular}
                \caption{Performances em diferentes PCs}
            \end{table}

            \begin{table}[hbt!]
                \centering
                \begin{tabular}{|*{4}{c|}}
                    \hline
                    & \thead{PC 1}&\thead{PC 2}&\thead{PC 3}\\
                    \hline
                    tests 1.txt & - & - & - \\
                    tests 2.txt & - & - & - \\
                    \hline
                \end{tabular}
                \caption{Tempos de execução em diferentes PCs}
            \end{table}

        \subsection{Possíveis otimizações e futuras limitações}
        \lipsum[1]
    \section{Conclusão}
        Para concluir, acreditamos que o desenvolvimento desta primeira fase foi uma boa maneira de consolidar tanto os nosso conhecimentos de C com conhecimentos mais avançados relacionadas a gestão de memória, encapsulamento e modularidade. Apesar de tudo, a nosso ver ainda existe espaço para melhorias
        no uso destes conceitos e pretendemos desenvolve-los a um nível ainda mais avançado do que
        os apresentados na próxima fase do trabalho.
\end{document}